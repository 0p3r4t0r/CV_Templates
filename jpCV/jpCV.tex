\documentclass[a4paper,12pt]{jpCV}
\usepackage{hyperref}
\hypersetup{
    colorlinks=true,
    urlcolor=blue,
}
\definecolor{tablesec}{rgb}{0.75, 0.75, 0.75}
\definecolor{tablesubsec}{rgb}{0.85, 0.85, 0.85}
\usepackage{tikz}
\newcommand*\circled[1]{\tikz[baseline=(char.base)]{
    \node[shape=circle,draw,inner sep=2pt] (char) {#1};}}
\begin{document}

\begin{header}[令和元年年12月21日]
    \ruby{\textbf{Your}}{フリガナ}
    \hspace{0.5ex}
    \ruby{\textbf{Name}}{フリガナ}
    \vspace{2ex}
\end{header}

\photo{./recent_photo.png}

\begin{info}[1.25]
    \begin{tabular}{ | m{3cm}<{\hspace{0.5cm}} | m{9.0cm} |}
        \hline 
        生年月日  & 平成XX年XX月XX日  \\
        \hline
        現住所    & 〒123-4567 何処か\newline
                      アパート名部屋番号  \\ 
        \hline
        メールアドレス    & nannchyarakannchyara@example.com  \\ 
        \hline
        携帯電話  & 1-234-567-8910  \\
        \hline
    \end{tabular}
\end{info}

\begin{body1}[1.75]
    \begin{tabular}{ | Y | M | m{14.75cm} | }
        \hline
        \rowcolor{tablesec}
        年      & 月& \multicolumn{1}{c |}{学歴・職歴}      \\
        \hline
        \rowcolor{tablesubsec}
                &   & \multicolumn{1}{c |}{学歴}            \\	%Schooling
        \hline
        平成20  & 8 &  〇〇高校 \tab 入学                  \\
        \hline
        平成24  & 6 &  〇〇高校 \tab 卒業                   \\
        \hline
        平成24  & 8 &  〇〇大学物理学部物理学科 \tab 入学	\\
        \hline
        平成28  & 5 &  〇〇大学物理学部物理学科 \tab 卒業	\\
        \hline
        \rowcolor{tablesubsec}
                &   & \multicolumn{1}{c |}{職歴}		    \\	%Work History
        \hline
        平成25  &10& 〇〇大学〇〇研究室 入室\\
        \hline
                &&\tab 主にBash言語で遺伝子シミュレーションのデータ分析。\\
        \hline
                &&\tab VMDとCHARMMを用いて遺伝子シミュレーション。\\
        \hline
                &&\tab 平成30年11月Springer-Natureで論文出版:
                    \href{link-to-article}{link-to-article}\\
        \hline
        平成27  &9&転職より退室\\
        \hline
        平成30  &12& 〇〇 入社               \\ 
        \hline
                &&\tab PHPのウェブアップリケーションをpython-djangoに翻訳。\\
        \hline
                &&\tab 3年分のデータとユーザアカウント情報の移行。\\
        \hline
                &&\tab フロントエンドはBootstrapでの完全書き直し。\\
        \hline
        令和元年&06&一身上の都合により退職\\
        \hline
        平成30  &04& 〇〇大学附属中学校・高等学校 入社\\
        \hline
                &&\tab 英語で社会と情報、物理、サイエンス、数学、英語を担当しました。\\
        \hline
                &&\tab 日本語で数学と社会と情報を担当しました。\\
        \hline
                &&\tab 教材管理システム\href{link-to-site}{link-to-site}のデサインと開発。\\
        \hline
                &&\tab 現在に至る\\
        \hline
    \end{tabular}
\end{body1}

~\newpage

\begin{body2}[1.75]
    \begin{tabular}{ | Y | M | m{14.75cm} | }
        \hline
        \rowcolor{tablesec}
        年      & 月& \multicolumn{1}{c |}{免許・資格}      \\
        \hline
        平成30  &03& 中学校教論特別免許(英語)東京都教育委員会\\
        \hline
        平成30  &03& 高等学校教論特別免許(英語)東京都教育委員会\\
        \hline
        平成30  &03& 中学校教論特別免許(理科)東京都教育委員会\\
        \hline
        平成30  &03& 高等学校教論特別免許(理科)東京都教育委員会\\
        \hline
        平成30  &08& 日本語能力試験一級 \\
        \hline
    \end{tabular}
\end{body2}


\begin{body2}[1.3]
    \begin{tabular}{ | m{18.15cm} | }
        \hline
        \rowcolor{tablesec}
        \multicolumn{1}{ | c |}{志望動機} \\
        \hline
            大学時代に学費を稼ぐために様々なアルバイトをさせていただきました。
            そのアルバイトは\\
            プログラミング・IT系と教育系の仕事が多かったです。
            様々なアルバイトを経験しましたが、\\
            最終的には教育の道を選んで
            副業として教育関係のIT会社に務めることになりました。\\
            しかし、教育はあくまでも生徒に古い知識を覚えさせるか、
            知られている結論まで\\
            たどり着かせることが多いです。
            教員として実際に未だに蓄えた知識のもとに本当に\\
            新しいことを作り出せる機会が少ないと思います。
            もっと先端的な技術を学び、新しいことを\\
            作り出したい気持ちが有り転職活動のきっかけになりました。\\
            \\
        \hline
    \end{tabular}
\end{body2}

\begin{body2}[1.3]
    \begin{tabular}{ | m{18.15cm} | }
        \hline
        \rowcolor{tablesec}
        \multicolumn{1}{ | c |}{趣味・特徴} \\
        \hline
        趣味:ボルダリング、登山、スノーボード、バイオリン\\
        パソコンのOS:Arch Linux\\
        愛用してるプログラミング言語:Python\\
        \\
        \hline
    \end{tabular}
\end{body2}

\begin{body2}[1.75]
    \begin{tabular}{ | m{18.15cm} | }
        \hline
        \rowcolor{tablesec}
        \multicolumn{1}{ | c |}{本人希望欄} \\
        \hline
        アプリケーション開発を希望します。
        Linux系またはMacOSの開発環境でお願いいたします。\\
        \\
        \hline
    \end{tabular}
\end{body2}

\begin{body2}[1.75]
    \begin{tabular}{ | m{8.85cm} | m{8.85cm} | }
        \hline
            通勤時間:\tab 約\tab 1時間\tab 分&
            扶養家族(配偶者を除く);\tab 0人\\
        \hline
            配偶者:\tab \tab 有 \tab \tab \circled{無}&
            配偶者の扶養義務:\tab 有 \tab \tab \circled{無}\\
        \hline
    \end{tabular}
\end{body2}
\end{document}

